\documentclass{article}
\usepackage{cite}
\usepackage{graphicx} % Required for inserting images
\usepackage{mathtools}
\usepackage[linesnumbered,ruled,vlined]{algorithm2e}
\usepackage[a4paper, total={6in, 8in}]{geometry}
\usepackage{hyperref}
\usepackage{amsmath}
\usepackage{amsthm}
\newtheorem{theorem}{Theorem}[section]
\newtheorem{lemma}[theorem]{Lemma}
\usepackage{enumitem}

\title{Group 2: Assignment 1.1}
\author{Niv Adam, David Kaufmann, Casper Kristiansson, Nicole Wijkman}
\date{\today}

\begin{document}

\maketitle

\section{B-Level Problem (1.2)}
Let $B = \{b_1,..,b_n\}$ and $C = \{c_1,...,c_m\}$ be sets containing buildings and customers as defined in the task description. We define the following algorithm.
\begin{algorithm}
\caption{B-level problem}
\DontPrintSemicolon
\SetKwFunction{FMain}{kSuppliers}
\SetKwProg{Fn}{Function}{:}{}
    \Fn{\FMain{$k$, $B$, $C$}}{
        Let $b\in B$ be the building, so $\exists c\in C: d(b,c)\leq d(b',c') \forall b'\in B,c'\in C$\;
        Let $S=\{b\}$\;
        \While{$|S|<k$}{
            Let $c_i\in C$ be such that $d(c_i,S)$ is maximized\;
            Let $b_j \in B$ be such that $d(b_j,c_i)$ is minimized\;
            $S\leftarrow S \cup \{b_j\}$ \;
        }
        \Return{S}
    }
\end{algorithm}

\subsection{Approximation Ratio}
Let $S_{Opt}$ be the optimal set of buildings. $S_{Opt}$ partitions $C$ into $k$ clusters. Customer $c_i$ belongs to cluster $x$ if $x$ is the point in $S_{Opt}$ closest to $c_i$. Additionally, we say $b_j$ lays within cluster $x$ if $d(b_j,x) \leq Opt$. With this we can define the following lemmas.

\begin{lemma}\label{lma:SameCluster}
    If $x,y$ lay within the same cluster $d(x,y) \leq 2 Opt$
\end{lemma}
\begin{proof}
    See lecture slides Lecture 3 slide 45 \cite{Lecture3}.
\end{proof}
Note that \ref{lma:SameCluster} is true for all points within the cluster. Independent of whether they are buildings or customers.

\begin{lemma}\label{lma:BuildingCustomer}
    If the algorithm selects building $b$, a customer $c$ exists such that $d(b,c) \leq Opt$.
\end{lemma}
\begin{proof}
    We can assume customer $c$ is the closest to $b$ from all customers. Then $c$ must have been the customer furthest away from $S_{Alg}$ before adding $b$. We also know that customer $c$ belongs to some cluster $x$. Let us assume $d(b,c) > Opt$ and contradict it.
    \begin{enumerate}[label=\textbf{Case \arabic*:}, left=0pt, itemindent=*, labelindent=1em]
        \item $b \in S_{Opt}$. Then customer $c$ belongs to cluster $b$ since it is the closest building. But then the optimal solution can not be $Opt$ since $d(b,c) > Opt$.
        \item $b \notin S_{Opt}$. The optimal solution can't be $Opt$ either, since $$\forall b_x\in B:d(b_x,c) \geq d(b,c) > Opt$$
    \end{enumerate}
    Therefore the assumption $d(b,c)>Opt$ must be wrong.
\end{proof}
Note that this also is true for the first selected building, since $$Opt = \max_{i=1..m} d(c_i,S_{Opt}) \geq \min_{\substack{i=1..m,\\j=1..n}} d(c_i, b_j)$$

\begin{lemma}\label{lma:onePerCluster}
    If the algorithm picks one building from within each cluster then $Alg \leq 2 Opt$.
\end{lemma}
\begin{proof}
    See lecture slides Lecture 3 slide 47 \cite{Lecture3}.
\end{proof}

\begin{lemma}\label{lma:pickSameCluster}
    If the algorithm selects both $b_i$ and $b_j$ from within the same cluster $x$ then $Alg \leq d(b_i,b_j)+d(b_j,c)$
\end{lemma}
\begin{proof}
    Suppose $b_i$ is picked first, then $b_j$. When $b_j$ is picked, it is the building closest to the customer $c$ that is furthest away from all previously picked buildings.
    \begin{enumerate}[label=\textbf{Case \arabic*:}, left=0pt, itemindent=*, labelindent=1em]
        \item $c\in X$ then using Lemma \ref{lma:SameCluster} $Alg\leq d(b_i,c)\leq 2Opt$
        \item $c\not\in X$ then using Lemma \ref{lma:BuildingCustomer} $Alg \leq d(b_i,c) \leq d(b_i,b_j)+d(b_j,c)\leq 3Opt$
    \end{enumerate}
    So we have $Alg=\max_{k=1,...,n} d(b_k,S_{Alg})\leq d(b_i,b_j)$.
\end{proof}
\ \\
\ \\
\ \\
\noindent
\textbf{Conclusion} We are left with the following cases of how the buildings we choose with the algorithm are spread across the clusters:
\begin{enumerate}[label=\textbf{Case \arabic*:}, left=0pt, itemindent=*, labelindent=1em]
    \item We pick two buildings $b_i,b_j$ from within the same cluster. With Lemma \ref{lma:pickSameCluster} (using \ref{lma:SameCluster} and \ref{lma:BuildingCustomer}) follows that $Alg\leq 3Opt$.
    \item  We pick a building $b$ that is not within any cluster. Lemma \ref{lma:BuildingCustomer} $\Rightarrow$ customer $c$ exists with $d(c,b) \leq Opt$. Lemma \ref{lma:SameCluster} $\Rightarrow$ all customers $c_y$ in the same cluster as $c$ have a distance lower than $d(b,c) + d(c,c_y) \leq 3Opt$ to $b$. Therefore $Alg \leq 3Opt$
    \item We pick one building from each cluster. Lemma \ref{lma:onePerCluster} $\Rightarrow Alg \leq 2Opt \leq 3Opt$ 
\end{enumerate}
This in total gives $Alg \leq 3Opt$.

\subsection{Time Analysis}
\subsubsection{Steps}
\begin{enumerate}
    \item Finding the shortest distance between a customer and a building requires comparing all buildings from set $B$  with the distance of all customers from set $C$. This will have a time complexity of $O(|B|\cdot|C|)$.
    \item The while loop will then run for $k-1$ iterations since we have initialized $S$ with one building. The algorithm will do the following in each iteration:
    \begin{enumerate}
        \item[•] Find the maximum distance between a customer and a set $S$ of buildings which will take $O(k\cdot|C|)$.
        \item[•] Check the closest building to it which takes $O(|B|)$
    \end{enumerate}
    Therefore, each iteration of the while loop will take $O(k\cdot|C|+|B|)$.
\end{enumerate}

\subsubsection{Overall Time Complexity}
The overall time complexity is the sum of the initialization step and the time spent in the while loop.
\\
\\
Time complexity $= O(|B|\cdot|C|+(k-1)(k\cdot|C|+|B|))\rightarrow O(|B|\cdot|C|)+O(k^2\cdot|C| + k\cdot|B|))$
\\
\\
It follows that the time complexity is in $O(|B|^2 \cdot |C|)$


\bibliographystyle{IEEEtran}
\bibliography{main}

\end{document}
