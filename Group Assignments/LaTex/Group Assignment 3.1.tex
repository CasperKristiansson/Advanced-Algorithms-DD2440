\documentclass{article}
\usepackage{cite}
\usepackage{graphicx} % Required for inserting images
\usepackage{mathtools}
\usepackage[linesnumbered,ruled,vlined]{algorithm2e}
\usepackage[a4paper, total={6in, 8in}]{geometry}
\usepackage{hyperref}
\usepackage{amsmath}
\usepackage{amssymb}
\usepackage{amsthm}
\usepackage{xcolor}
\newtheorem{theorem}{Theorem}[section]
\newtheorem{lemma}[theorem]{Lemma}
\usepackage{enumitem}

\SetKwInOut{KwData}{Input}
\SetKwInOut{KwResult}{Output}
\SetKw{KwOr}{or}

\title{Group Assignment 3.1}
\author{Niv Adam, David Kaufmann, Casper Kristiansson, Nicole Wijkman}
\date{\today}

\begin{document}

\maketitle

\setlength\parindent{0pt}   % Disable paragraph indent.
\setlength{\parskip}{\bigskipamount}    %Newline after each paragraph

\section{E-Level Problem (3.1)}
The objective is to create an algorithm that can verify that the sum of \(n\)-bit integers from \(m\) computers equals an \(n\)-bit integer. This means we want to confirm if the sum \(\sum_{i=1}^{m} x_i = y\) holds true where \(x_i\) is the \(n\)-bit number from the \(i\)-th computer. The constraint is that the solution with a probability of at least \(0.9\) is the sum of the integers of the computers with a complexity that requires \(O(m\log^{10}n)\) bits of communication.

To achieve the required complexity of \(O(m\log^{10}n)\), we employ a fingerprinting technique based on the selection of primes. The algorithm leverages the Prime Number Theorem, which allows us to make probabilistic statements about the equality of fingerprints \(h_p(x)\) and \(h_p(y)\), where \(h_p\) is the fingerprint function associated with a prime \(p\).

The fingerprinting algorithm involves the selection of a prime \(p\). 

Based on the Prime Number Theorem we can determine that if \(x \neq y\), then the probability of the fingerprint functions \(h_p(x)\) and \(h_p(y)\) being equal, denoted as \(\Pr[h_p(x) = h_p(y)]\), is bounded by \(\frac{1}{2}\). This probabilistic property forms the foundation of our algorithm and provides a means to distinguish between different inputs.

\[
\Pr[h_p(x) = h_p(y)] \leq \frac{1}{2}
\]

We then need to establish a bound for the prime so that we can sample specific primes within a range of [2, M] where M is a value large enough to ensure a low probability of collision. This means that for example by setting \(M = K \times n \times ln(n),\) where \(K\) ... % Got stuck

% Do we really need to show the entire theorem and proof for the prime selection?

Let us look at the $L$ different fingerprints the server received from each computer. For each prime $p\in L$ it holds that
$$((x_1 \mod p)+(x_2 \mod p) + ...)\mod p = (x_1 + x_2 + ...)\mod p$$
Then 
$$((x_1 \mod p)+(x_2 \mod p) + ...)\mod p = y\mod p$$
holds if $x_1+x_2+...=y$ holds. In case it does not, we know from the theorem in the lecture that the probability $Pr[$$((x_1 \mod p)+(x_2 \mod p) + ...)\mod p = (x_1 + x_2 + ...)\mod p]\leq 0.5$.

% \bibliographystyle{IEEEtran}
% \bibliography{main}


\end{document}